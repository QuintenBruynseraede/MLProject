\documentclass[10pt,a4paper]{article}
\usepackage[utf8]{inputenc}
\usepackage{amsmath}
\usepackage{amsfonts}
\usepackage{amssymb}
\usepackage{tcolorbox}
\usepackage[numbers]{natbib}
\usepackage[margin=1.5in]{geometry}

\title{Machine Learning Project: Report 2}
\author{Ignace Bleukx \and Quinten Bruynseraede}
\begin{document}
\maketitle
\section{Introduction}
\subsection{Evaluation metrics}
Evaluation of agents is traditionally done using \textbf{NashConv} and \textbf{exploitability}. We introduce these concepts here, using terminology consistent with 

Given a policy $\pi$,
\subsection{Algorithm 1: Fictitious Self-Play}
\subsubsection{Extension: Neural Fictitious Self-Play}
\subsection{Algorithm 2: Counterfactual Regret Minimization}
\subsubsection{Extension: Regression Counterfactual Regret Minimization}
\subsubsection{Extension: Counterfactual Regret Minimization against best responder}
\subsubsection{Extension: Deep Counterfactual Regret Minimization}


\section{Kuhn Poker}
\begin{tcolorbox}
\begin{itemize}
\item{Which algorithm is most suitable to develop an agent to play Kuhn Poker, minimizing exploitability?}
\item{Can we exploit properties of Kuhn Poker to optimize parameters?}
\end{itemize}
\end{tcolorbox}

\section{Leduc Poker}
\begin{tcolorbox}
\begin{itemize}
\item{Which algorithm is most suitable to develop an agent to play Leduc Poker, minimizing exploitability?}
\item{Can we exploit properties of Leduc Poker to optimize parameters?}
\item{Can we combine agents into an ensemble that minimizes exploitability further than its parts?}
\end{itemize}
\end{tcolorbox}


\bibliographystyle{plainnat}
\bibliography{lit}
\section*{Appendix}
\subsection{Time spent}
\end{document}

CFR
Regression CFR
CFR-BR
Deep CFR